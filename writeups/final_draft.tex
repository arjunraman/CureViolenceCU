\documentclass[11pt,preprint]{aastex}
 %\documentclass[12pt]{emulateapj}
\usepackage[margin= 1.0in]{geometry}    % See geometry.pdf to learn the layout options. There are lots.
\geometry{letterpaper} % or letter or a5paper or ... etc
\usepackage{float}
\usepackage{amssymb,amsmath}
\usepackage[]{epsfig,graphicx}
\usepackage{color}
\usepackage{verbatim}
\DeclareGraphicsRule{.tif}{png}{.jpg}{`convert Num1 `dirname Num1`/`basename Num1 .tif`.jpg}
\newcommand{\units}[1]{\ensuremath{\, \mathrm{#1}}}
\usepackage{enumitem}
\usepackage{natbib}
\newcommand{\degree}{\ensuremath{^\circ}}
\usepackage[maxfloats=25]{morefloats}
\usepackage[normalem]{ulem}
\usepackage{hyperref}
\usepackage{ amssymb }
\usepackage{lscape}
\newcommand{\TRANSPOSE}{\ensuremath{T}}

\bibliographystyle{apalike}


\begin{document}
\title{Cure Violence: Telling the Story of Gun Violence}

 \author{Andy Enkeboll, Erin Grand, Mayank Misra}
 \author{In partnership with Booz Allen Hamilton}
 \affil{Data Science Institute, Columbia University, New York, NY 10027}
 
\date{\today}             

\tableofcontents

\section{Introduction}
Cure Violence is a non-profit that focuses on stopping the spread of  gun violence in communities by using the methods and strategies associated with disease control, detecting and interrupting conflicts, identifying and treating the highest risk individuals, and changing social norms. Specifically, they focus on:
\begin{enumerate}
\item Detecting and interrupting potentially violent events by preventing retaliations and mediating conflicts through on-the-ground efforts within communities.
\item Assessing high risk candidates, changing behavior, and providing treatment through 1-on-1 case work.
\item Engaging the community to change by responding to every shooting, and organizing community events and people.
\end{enumerate}

Through partners at Booz Allen Hamilton, Cure Violence is seeking to make better use of the data they collect and the independent reports that have been written about them.  From the outset, Cure Violence have outlined four main goals they have for this capstone:
\begin{enumerate}
    \item Demonstrate how Cure Violence unique operating model can be deployed across different geographies to reduce violence and increase cost savings to stakeholders both direct and indirect within communities and cities of Cure Violence operations.
    \item Data analysis and/or model to inform growth of existing program services
    \item Data to inform potential strategic partners \& alliances
\end{enumerate}


\section{Plan / Milestones}
From the stated goals outlined above, a plan was created mid-semester that more accurately represents what this Capstone project can do in terms of time and people:

\begin{enumerate}
\item Cost benefit analysis - simple:
How many Incidents were avoided, multiplied by the cost saving scalar,
Extract data from public reports
\item Cost benefit analysis - complex:
Comparison to national averages in Chicago/Baltimore,
Exploration of the government data,
Lift of incidents avoided
\item Joining and consolidating internal and external government data sets
\item Solid path forward for cure violence of where to go next from a cost benefit analysis perspective
\end{enumerate}

As the semester neared its end, the milestones were reduced to focus on two specific areas:
\begin{enumerate}
    \item Cost-benefit analysis
    \begin{enumerate}
        \item How many incidents were avoided, multiplied by the cost saving scalar
        \item Extracted data from reports
        \item Extracted model from the Mother Jones analysis
    \end{enumerate}
    \item Visualization of Crime Reports
    \begin{enumerate}
        \item Filter based on type of violence
        \item Allows for decisions on where to go next
        \item Incorporate cost-benefit analysis results
        \item Ability to “drop in” mediators to view their effects in real time
    \end{enumerate}
\end{enumerate}

\section{Data Collections and Analysis}
Data was acquired from two sources:
\begin{enumerate}
    \item Public record paper results from independent researchers in Chicago and Baltimore. 
    \item City and state open data, largely powered by the Socrata OpenData initiative, for the cities of Chicago, Baltimore and New York.
\end{enumerate}

\subsection*{Public Reports}
Three reports have been conducted by researchers in Chicago and Baltimore, at John's Hopkins,  University of Illinois: Chicago and Northwest University. Each study looked at how the Cure Violence programs were used and assessed the success of the programs. It was found that the Cure Violence framework was applied in Chicago and Baltimore neighborhoods with measurable success.   


\subsubsection*{Johns Hopkins University: Baltimore, 2012}

The John Hopkins team looked at research from two programs in Baltimore, (a) and (b). for each program there were 30-40 participants taking surveys and recording internal data. 

Data from the Baltimore Police Department for homicides and nonfatal shootings from January 1, 2003 to December 31, 2010
Surveys of program participants from 2007 - 2010


Three of the four program sites experienced large, statistically significant, program related reductions in homicides or nonfatal shootings without having a counter-balancing significant increase in one of these outcome measures.


\subsubsection*{UI Chicago: Chicago, 2009, Northwestern University: Chicago, 2014}

Raw crime counts showed a 31\% reduction in homicide, a 7\% reduction in total violent crime, and a 19\% reduction in shootings in the targeted districts. These effects are significantly greater than the effects expected given the declining trends in crime in the city as a whole.

The reduced levels of total violent crime, shootings, and homicides remained constant past the time frame of the survey analysis.  The effects of the intervention were seen immediately and therefore it is unlikely that effects were only due to increased police activity.

\subsection*{OpenData}
The data provided by the OpenData initiatives in each of these cities made it very straightforward to drive our recommendation model.  Gun related crime reported provide location information, and each of the cities investigated provide many years of data to work with. Census/population data can be added in as well to determine how similar demographics of cities might be.
\newpage
\subsection*{Proprietary Data}
Cure Violence itself keeps careful data from each intervention. There are:
\begin{enumerate}
\item Case notes (interventions)
\item Conflict mediation reports
\item Community Events \& participants
\item Crime data reported specifically in the bounds of the neighborhood
\end{enumerate}
Extensions of this project will be able to incorporate these data sets, allowing Cure Violence to extend their reach even further.

\section{Model of Cost Savings}
The evaluations of Baltimore and Chicago focused on the correlation between application of the model and related killing and shooting incidences in the area. One of the key objectives of our analysis is to quantify this impact in financial terms.  

In an effort to estimate the dollar impact from gun crimes, we have used the cost categories and estimates described in Mother Jones analysis done by Mark Follman, Jeah Lee and Julia Lurie on the true cost of gun violence in America as a template to create a financial model.  This analysis was based on the research done by economist Ted Miller of Pacific Institute for Research and Evaluation.  
\begin{figure}[h]
\centering
\includegraphics[width=6in]{CureViolenceCostModel.png}
\caption{Example Cost Model}
\end{figure}

\subsection*{Model Assumptions}
Our intent with the cost model is to provide an estimate on the dollar impact to the societies where the Cure Violence model was applied.  We made the following decisions to tie the three primary sources on which our cost estimation model is based:  

To keep the model simple, we have rolled up cost categories as described in the Mother Jones article.  (1)	For direct costs, the 'Police' and 'Legal service and adjudication' categories have been clubbed under 'Police and Legal'.  The 'Medical' and 'Emergency transport' categories are aggregated under 'Medical care and Transport' in our model. (2)	For indirect costs, 'work cost of victims and perpetrators' and 'cost to employers' are aggregated under 'lost wages and productivity'.

We recognize that the way gun crime incidences are defined in the North Western, John Hopkins, and the Mother Jones studies will not match perfectly.


The estimate cost for a particular category were derived, either directly as quoted in the Mother Jones article, or have been deciphered from the graphs accompanying the write up. 

\section{Visualization}
To better recognize the patterns of violence over time in a city, we launched a web tool that visualizes geographic "hot spots" for a given data set (city).  The tool provides the option to filter out different types of gun violence as well as drop in mediators to view the tangible effects of Cure Violence practices in a neighborhood.

\begin{figure}[h]
\centering
\includegraphics[width=6in]{Cure Violence- Final Presentation.png}
\caption{Visualization screenshot}
\end{figure}

\newpage
\section{Suggested Steps  for Advancing Value from Data for CV}
In order to further advance the value from data to impact core functions of Cure Violence, additional work should be done on this project.  We have built a model and a tool that can be handed over to another group to implement what we weren't give a chance to implement.  These things include:
\begin{enumerate}
    \item Tie in even more data sets: demographics, quality of life, internet usage, etc to build a growth model that can be applied across cities
    \item Leverage Web MVP
    \begin{enumerate}
        \item Add additional data (layering / pockets)
        \item Create functionality for prioritization of local areas
        \item Expand geographies filter
    \end{enumerate}
    \item Embed a easy to use customer view within Website.  Consider optimizing website with something interactive that customers can play with, and make it extensible to other cities
    \item Incorporate data from internal data sets be synced with a web presence: show areas of mediator presence and events to highlight impact

\end{enumerate}
\end{document} 
